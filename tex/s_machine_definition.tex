In this section we introduce a nondeterministic Turing machine model called the sweep machine of S-machine.
The S-machine performs its computations during successive sweeps of the tape.
The $i$th sweep, $i > 1$ consists of $i$ left moves followed by $(i+1)$ right moves.
The leftmost symbol on the tape is a special end marker symbol, which also signals the reversal of a sweep.

The S-machine operates with sets of tape symbols and sets of states.
At any step of computation, the machine state would consist of a set of states and scanned cell
would contain a set of tape symbols.
The set of next states of the machine and the set of tape symbols to be written in the scanned
cell by the use of a choice function.
This choice function decides how the current states of the machine are to be
paired with the symbols in the scanned cell.
There are finitely many possible choice function values, each choice function is also a finite function.
A computation of the S-machine consisting of $k$ sweeps is completely specified by a choice
function sequence of $k^2 + 2k$.
The S-machine is an acceptor model.
An input is accepted by this machine if and only if there exists a choice function sequence that
results in the machine entering an accepting state at the end of $k$ sweeps, for some
$k$.
If the input is not accepted by the machine, then the sequence of sweeps could
be continued indefinitely, with all possible choice function sequences, without $M$ 
ever entering an accepting state.
Formal definition of $M$ follows.

An \emph{S-machine} $M$ is a sixtuple 
$M = \langle Q, \Sigma, \Gamma, \delta, \{q_0\}, H \rangle$.
where $Q$ is a finite set of \emph{states}, and 
$Q = Q^L \bigcup Q^R \bigcup P^R \bigcup \{q_0\}$.
$Q^L$ is a set of \emph{left moving states}, $Q^R, P^R$ are 
sets of \emph{right moving} states and $\{q_0\}$ a designated 
set of \emph{start states}.
All of these sets of states are mutually disjoint.


$\Sigma$ is a finite set of input symbols, excluding the symbols $\cancelb$, $\cancelc$, $z$.
$\Gamma$ is a finite set of \emph{tape symbols}.
Each element of $\Gamma$ is an ordered pair
$<x,y>$ of symbols:
\begin{equation*}
    \Gamma = \left\{ <x,y> | x \in \Sigma \bigcup \{\cancelb\},
    y \in \{\cancelb, z\}\right\} \bigcup \left\{<\cancelc, \cancelc>\right\}
\end{equation*}

$H \subseteq Q$ is a set of \emph{accepting states}.
Properties of $H$ will be discussed later.
The transition function $\delta$ is defined by the instruction formats listed below.
The superscripts $L$ and $R$ are used to denote membership in $Q^L$ and $Q^R \bigcup P^R$.
$L$ denotes a left move and $R$ is a right move.
\begin{enumerate}[label=(\roman*)]
    \item \label{itm:start} For each $q_0 \in \{q_0\}, <x,z> \in \Gamma$ there exists
    $<x_1,z>,\ldots,<x_k,z> \in \Gamma$ and $q_{i_1}^L, \ldots, q_{i_k}^L \in Q_L$ 
    such that
    $(q_0, <x,z>) \mapsto \left\{(q_{i_1}^{L}, <x_1,z>), \ldots, (q_{i_k}^L, <x_k,z>)\right\}$
    is in $\delta$.
    \item \label{itm:sweepleft} For each $q^L \in Q^L, <x,\cancelb> \in \Gamma,$ there exists
    $<x_1,\cancelb>,\dots,<x_k,\cancelb> \in \Gamma$ and
    $q_{i_1}^L,\dots,q_{i_k}^L \in Q^L$ such that
    $(q^L, <x,\cancelb>) \mapsto \left\{(q_{i_1}^L,<x_1,\cancelb>),\dots,(q_{i_k}^L, <x_l,\cancelb>)\right\}$ is in $\delta$.
    \item \label{itm:leftreverse} For each $q^L \in Q^L$, there exists $q^R \in Q^R$ such that
    $(q^L, <\cancelc,\cancelc>) \mapsto (q^R,<\cancelc,\cancelc>)$  is in $\delta$.
    \item \label{itm:sweepright} For each $q^R \in Q^R, <x,\cancelb> \in \Gamma$, there exists
    $q_{i_1}^R,\ldots,q_{i_k}^R \in Q^R,
    <x_1,\cancelb>,\ldots,<x_k,\cancelb> \in \Gamma$ such that
    $(q^R,<x,\cancelb>) \mapsto \left\{(q_{i_1}^R, <x_1,\cancelb>)\ldots
    (q_{i_k}^R, <x_k,\cancelb>)\right\}$ is in $\delta$.
    \item \label{itm:extend} For each $q^R \in Q^R, <x,z> \in \Gamma$, there exists
    $p_{i_1}^R,\ldots,p_{i_k}^R \in P^R$ and 
    $<x_1\cancelb>,\ldots,<x_k,\cancelb> \in \Gamma$ such that
    $(q^R, <x,z>) \mapsto \left\{p_{i_1}^R,<x_1,\cancelb>),\ldots,(p_{i_k}^R, <x_k,\cancelb>)\right\}$ is in $\delta$.
    \item \label{itm:turnaround} For each $p^R \in P^R$ and $<x,\cancelb> \in \Gamma$, there exists
    $q_{i_1}^L,\ldots,q_{i_k}^L \in Q^L$ and 
    $<x_1,z>,\ldots,<x_k,z> \in \Gamma$ such that
    $(p_i^R,<x,\cancelb>) \longrightarrow \left\{(q_{i_1}^L,<x_1,z>),\ldots,(q_{i_k}^L,<x_k,z>)\right\} $
    is in $\delta$.
\end{enumerate}


We shall consider $\delta$ to be the smallest set of instructions closed under rules~\ref{itm:start}
to~\ref{itm:turnaround} above.
We shall assume that once the machine $M$ is in a halting state, all subsequent
states it enters into are halting states.
The symbol $\cancelc$ is used as a left end marker.
The symbol $z$ which only appears as the second component of a tape symbol is called a sweep marker.
It marks the cell at which the next sweep will begin.
The symbol $\cancelb$ plays a special role.
We assume that $M$ does not enter a halt state on scanning a $<\cancelb,\cancelb>$ or $<\cancelb,z>$ tape symbol.
The machine $M$ performs all of its computations in terms of sets.
At any stage of computation, each cell of the tape of $M$ contains a (finite) set of tape symbols.
The state of the machine $M$ is a subset of left moving or right moving states.
Each step of computation of $M$ is assumed to consist of the following sequence of operations:
\begin{enumerate}[label=\alph*)]
    \item \label{itm:scan} scan the symbols in the cell
    \item \label{itm:write} write new symbols
    \item \label{itm:state} change states
    \item \label{itm:move} move one cell to the left or right.
\end{enumerate}
According to the description above, the direction of move in~\ref{itm:move} is known.


\begin{aside}
There is a lot to unpack here, and the original paper is pretty terse.
So let's start by looking at the various pieces of the formal definition.

First, the structure of each cell in the tape consists of a set of tape symbols.
I.e., the contents of cell $c(i) \in \Gamma^*$, the power set of 
tape symbols.
We will see that only a restricted set of these tape symbols is
permissible in a moment.

The state of the S-machine will be some $q \in Q$.
This state can be one of four special types of state:
\begin{itemize}
\item $q \in Q^L$ : this is a left moving state, meaning that when the S-machine
advances, it will move left.
\item $q \in {q_0}$ : these are start states.
The S-machine will start in one of these states.
Per~\ref{itm:start}, the machine will start moving left, provided that the
marker for the start state is blank.
\item $q \in Q^R$ : this is a right moving state, meaning that when the S-machine
advances, it will move right.
\item $p \in P^R$ : this is also a set of right moving states, but they have some
additional properties.
From~\ref{itm:extend}, it follows that if the S-machine
is in the state $p \in P^R$, and it reads a cell consisting of an input symbol $x$,
and a blank marker $\cancelb$, then it will write the $z$ symbol in the marker position,
and transition to a left moving state.
\end{itemize}


Now, the set of properties laid out above suggest that not all tape symbol sets
are actually permissible.
In particular, the state of the marker part of the
tape symbol must be synchronized.
Let's revisit the rules in a bit more detail.

\begin{enumerate}[label=(\roman*)]
    \item If the S-machine is in an initial state $q_0$, the tape symbol active
    has some symbol $x$ and a blank marker $z$, then the machine will leave
    the blank marker alone, and enter a set of left moving state.

\[
\begin{aligned}
\begin{tikzpicture}[every node/.style={block},
        block/.style={minimum height=3.5em,outer sep=0pt,draw,rectangle,node distance=0pt}]
   \node (A) {$\,\substack{x\\\cancelb}\,$};
   \node (B) [left=of A] {$\ldots$};
   \node (D) [right=of A] {$\ldots$};
   \node (F) [above = 0.75cm of A,draw=blue,thick] {$q_0$};
   \draw[-latex] (F) -- (A);
   \draw (B.north west) -- ++(-1cm,0) (B.south west) -- ++ (-1cm,0) 
                 (D.north east) -- ++(1cm,0) (D.south east) -- ++ (1cm,0);
\end{tikzpicture}
\end{aligned}
\quad \longrightarrow \quad
\begin{aligned}
\begin{tikzpicture}[every node/.style={block},
        block/.style={minimum height=3.5em,outer sep=0pt,draw,rectangle,node distance=0pt}]
   \node (A) {$\,\substack{x\\\cancelb}\,$};
   \node (B) [left=of A] {$\ldots$};
   \node (D) [right=of A] {$\ldots$};
   \node (F) [above = 0.75cm of A,draw=blue,thick] {${q_1^L,\ldots,q_k^L}$};
   \draw[-latex] (F) -- (A);
   \draw[-latex,red] ($(F.west)!0.5!(B.east)$) -- ++(-7mm,0);
   \draw (B.north west) -- ++(-1cm,0) (B.south west) -- ++ (-1cm,0) 
                 (D.north east) -- ++(1cm,0) (D.south east) -- ++ (1cm,0);
\end{tikzpicture}
\end{aligned}
\]

    \item If the S-machine is in a left moving state, and the marker is $\cancelb$,
    then if will remain in a left moving set of states.
    In other words, if the cell is marked with $\cancelb$, the S-machine is guaranteed to move left, and
    leave the marker as $\cancelb$.  

\[
\begin{aligned}
\begin{tikzpicture}[every node/.style={block},
        block/.style={minimum height=3.5em,outer sep=0pt,draw,rectangle,node distance=0pt}]
   \node (A) {$\,\substack{x\\\cancelb}\,$};
   \node (B) [left=of A] {$\ldots$};
   \node (D) [right=of A] {$\ldots$};
   \node (F) [above = 0.75cm of A,draw=blue,thick] {$q_L$};
   \draw[-latex] (F) -- (A);
   \draw[-latex,red] ($(F.west)!0.5!(B.east)$) -- ++(-7mm,0);
   \draw (B.north west) -- ++(-1cm,0) (B.south west) -- ++ (-1cm,0) 
                 (D.north east) -- ++(1cm,0) (D.south east) -- ++ (1cm,0);
\end{tikzpicture}
\end{aligned}
\quad \longrightarrow \quad
\begin{aligned}
\begin{tikzpicture}[every node/.style={block},
        block/.style={minimum height=3.5em,outer sep=0pt,draw,rectangle,node distance=0pt}]
   \node (A) {$\,\substack{x\\\cancelb}\,$};
   \node (B) [left=of A] {$\ldots$};
   \node (D) [right=of A] {$\ldots$};
   \node (F) [above = 0.75cm of A,draw=blue,thick] {${q_1^L,\ldots,q_k^L}$};
   \draw[-latex] (F) -- (A);
   \draw[-latex,red] ($(F.west)!0.5!(B.east)$) -- ++(-7mm,0);
   \draw (B.north west) -- ++(-1cm,0) (B.south west) -- ++ (-1cm,0) 
                 (D.north east) -- ++(1cm,0) (D.south east) -- ++ (1cm,0);
\end{tikzpicture}
\end{aligned}
\]


    \item If the S-machine is in a left moving state, and it encounters $<\cancelc, \cancelc>$,
    it will enter a right moving state, and leave the $\cancelc$ marker in place.

\[
\begin{aligned}
\begin{tikzpicture}[every node/.style={block},
        block/.style={minimum height=3.5em,outer sep=0pt,draw,rectangle,node distance=0pt}]
   \node (A) {$\,\substack{\cancelc\\\cancelc}\,$};
   \node (B) [left=of A] {$\ldots$};
   \node (D) [right=of A] {$\ldots$};
   \node (F) [above = 0.75cm of A,draw=blue,thick] {$q^L$};
   \draw[-latex] (F) -- (A);
   \draw[-latex,red] ($(F.west)!0.5!(B.east)$) -- ++(-7mm,0);
   \draw (B.north west) -- ++(-1cm,0) (B.south west) -- ++ (-1cm,0) 
                 (D.north east) -- ++(1cm,0) (D.south east) -- ++ (1cm,0);
\end{tikzpicture}
\end{aligned}
\quad \longrightarrow \quad
\begin{aligned}
\begin{tikzpicture}[every node/.style={block},
        block/.style={minimum height=3.5em,outer sep=0pt,draw,rectangle,node distance=0pt}]
   \node (A) {$\,\substack{\cancelc\\\cancelc}\,$};
   \node (B) [left=of A] {$\ldots$};
   \node (D) [right=of A] {$\ldots$};
   \node (F) [above = 0.75cm of A,draw=blue,thick] {$q^R$};
   \draw[-latex] (F) -- (A);
   \draw[-latex,red] ($(F.east)!0.5!(D.east)$) -- ++(7mm,0);
   \draw (B.north west) -- ++(-1cm,0) (B.south west) -- ++ (-1cm,0) 
                 (D.north east) -- ++(1cm,0) (D.south east) -- ++ (1cm,0);
\end{tikzpicture}
\end{aligned}
\]


    \item If the S-machine is in a right moving state, and the marker is $\cancelb$, 
    it will remain in a right moving set of states.

\[
\begin{aligned}
\begin{tikzpicture}[every node/.style={block},
        block/.style={minimum height=3.5em,outer sep=0pt,draw,rectangle,node distance=0pt}]
   \node (A) {$\,\substack{x\\\cancelb}\,$};
   \node (B) [left=of A] {$\ldots$};
   \node (D) [right=of A] {$\ldots$};
   \node (F) [above = 0.75cm of A,draw=blue,thick] {$q^R$};
   \draw[-latex] (F) -- (A);
   \draw[-latex,red] ($(F.east)!0.5!(D.east)$) -- ++(7mm,0);
   \draw (B.north west) -- ++(-1cm,0) (B.south west) -- ++ (-1cm,0) 
                 (D.north east) -- ++(1cm,0) (D.south east) -- ++ (1cm,0);
\end{tikzpicture}
\end{aligned}
\quad \longrightarrow \quad
\begin{aligned}
\begin{tikzpicture}[every node/.style={block},
        block/.style={minimum height=3.5em,outer sep=0pt,draw,rectangle,node distance=0pt}]
   \node (A) {$\,\substack{x\\\cancelb}\,$};
   \node (B) [left=of A] {$\ldots$};
   \node (D) [right=of A] {$\ldots$};
   \node (F) [above = 0.75cm of A,draw=blue,thick] {${q_1^R,\ldots,q_k^R}$};
   \draw[-latex] (F) -- (A);
   \draw[-latex,red] ($(F.east)!0.5!(D.east)$) -- ++(7mm,0);
   \draw (B.north west) -- ++(-1cm,0) (B.south west) -- ++ (-1cm,0) 
                 (D.north east) -- ++(1cm,0) (D.south east) -- ++ (1cm,0);
\end{tikzpicture}
\end{aligned}
\]


    \item If the S-machine is in a right moving state, and the marker is $z$ (blank), then
    it will enter a state in $P^R$ and mark the cell as $\cancelb$.

\[
\begin{aligned}
\begin{tikzpicture}[every node/.style={block},
        block/.style={minimum height=3.5em,outer sep=0pt,draw,rectangle,node distance=0pt}]
   \node (A) {$\,\substack{x\\z}\,$};
   \node (B) [left=of A] {$\ldots$};
   \node (D) [right=of A] {$\ldots$};
   \node (F) [above = 0.75cm of A,draw=blue,thick] {$q^R$};
   \draw[-latex] (F) -- (A);
   \draw[-latex,red] ($(F.east)!0.5!(D.east)$) -- ++(7mm,0);
   \draw (B.north west) -- ++(-1cm,0) (B.south west) -- ++ (-1cm,0) 
                 (D.north east) -- ++(1cm,0) (D.south east) -- ++ (1cm,0);
\end{tikzpicture}
\end{aligned}
\quad \longrightarrow \quad
\begin{aligned}
\begin{tikzpicture}[every node/.style={block},
        block/.style={minimum height=3.5em,outer sep=0pt,draw,rectangle,node distance=0pt}]
   \node (A) {$\,\substack{x\\\cancelb}\,$};
   \node (B) [left=of A] {$\ldots$};
   \node (D) [right=of A] {$\ldots$};
   \node (F) [above = 0.75cm of A,draw=blue,thick] {${p_1^R,\ldots,p_k^R}$};
   \draw[-latex] (F) -- (A);
   \draw[-latex,red] ($(F.east)!0.5!(D.east)$) -- ++(7mm,0);
   \draw (B.north west) -- ++(-1cm,0) (B.south west) -- ++ (-1cm,0) 
                 (D.north east) -- ++(1cm,0) (D.south east) -- ++ (1cm,0);
\end{tikzpicture}
\end{aligned}
\]


    \item If the s-machine is in state $p^R \ in P^R$, and the cell is marked with $\cancelb$,
    then it will mark the cell as $z$ (blank) and enter a left moving state.

 \[
\begin{aligned}
\begin{tikzpicture}[every node/.style={block},
        block/.style={minimum height=3.5em,outer sep=0pt,draw,rectangle,node distance=0pt}]
   \node (A) {$\,\substack{x\\\cancelb}\,$};
   \node (B) [left=of A] {$\ldots$};
   \node (D) [right=of A] {$\ldots$};
   \node (F) [above = 0.75cm of A,draw=blue,thick] {$p^R$};
   \draw[-latex] (F) -- (A);
   \draw[-latex,red] ($(F.east)!0.5!(D.east)$) -- ++(7mm,0);
   \draw (B.north west) -- ++(-1cm,0) (B.south west) -- ++ (-1cm,0) 
                 (D.north east) -- ++(1cm,0) (D.south east) -- ++ (1cm,0);
\end{tikzpicture}
\end{aligned}
\quad \longrightarrow \quad
\begin{aligned}
\begin{tikzpicture}[every node/.style={block},
        block/.style={minimum height=3.5em,outer sep=0pt,draw,rectangle,node distance=0pt}]
   \node (A) {$\,\substack{x\\z}\,$};
   \node (B) [left=of A] {$\ldots$};
   \node (D) [right=of A] {$\ldots$};
   \node (F) [above = 0.75cm of A,draw=blue,thick] {${q_1^L,\ldots,q_k^L}$};
   \draw[-latex] (F) -- (A);
   \draw[-latex,red] ($(F.west)!0.5!(B.east)$) -- ++(-7mm,0);
   \draw (B.north west) -- ++(-1cm,0) (B.south west) -- ++ (-1cm,0) 
                 (D.north east) -- ++(1cm,0) (D.south east) -- ++ (1cm,0);
\end{tikzpicture}
\end{aligned}
\]

\end{enumerate}

While stated above, it is important to realize that no matter how the machine operates,
all the set of states are synchronized with respect to their motion and the marker state of the cell.
So let us define first an augmented set of tape symbols that includes $\cancelb$:
\begin{equation*}
    \Sigma^+ = \Sigma \bigcup {\cancelb}
\end{equation*}
and let us also define a set of marker symbols 
\begin{equation*}
    Y = \{\cancelb, z\}
\end{equation*}

So it is more accurate to describe the contents of the tape cell as
\begin{equation*}
    \Gamma^* = \left\{
    \{<x_1,y>,\ldots,<x_k,y>\} | <x_i, y> \in \Sigma^+ \times Y \bigcup <\cancelc, \cancelc>
    \right\}
\end{equation*}

In words, this means each cell contains a 2-tuple, consisting of a set ${x_i}$ and a marker $y$, 
where
\begin{itemize}
    \item The set $x_i$ comes from either the input symbols or $\cancelb$.
    \item The marker $y$ is either $\cancelb$ or $z$.
    \item A special case symbol $<\cancelc, \cancelc>$, can appear, as can
    $<{\cancelc \ldots \cancelc}, \cancelc>$.
\end{itemize}

No other combination can appear.
It may be helpful to think of $\cancelb$ as a \verb|NULL| symbol.
\end{aside}


At any step $i$ of computation, $sc(i)$ will denote the index of the cell
scanned in~\ref{itm:scan} and $Ac(i)$ the set of
states at the end of~\ref{itm:state} above.
It follows that at step $i, i\geq 1$, the machine scans the
symbols of the cell $sc(i)$ in states $Ac(i-1)$.
In computing $Ac(i)$ from the symbols in cell
$sc(i)$ and the states $Ac(i-1)$, we use a choice function $\alpha$.
A particular choice function
$\alpha$ maps each state of $M$ to a subset of $\Gamma$.
Before defining this computation, we introduce
the notion of a product.

Let $A$ be the set of states and $B$ a set of tape symbols of $M$.
We introduce the following
operator $*$ called \emph{product}.
\begin{align*}
    A*B &= \left\{q|\exists x \in \Gamma, q_1 \in A, a_1 \in B \suchthat (q,x) \in \delta(q_1, a_1)\right\}\\
    B*A &= \left\{x|\exists q \in Q, q_1 \in A, a_1 \in B \suchthat (q_1,x) \in
    \delta(q_1, a_1)\right\}.
\end{align*}
\begin{error}
Typo?
Should be
\begin{align*}
    A*B &= \left\{q|\exists x \in \Gamma, q_1 \in A, a_1 \in B \suchthat (q,x) \in \delta(q_1, a_1)\right\}\\
    B*A &= \left\{x|\exists q \in Q, q_1 \in A, a_1 \in B \suchthat (q,x) \in
    \delta(q_1, a_1)\right\}.
\end{align*}
\end{error}
This $A*B$ is the set of all possible next states of $M$ and $B*A$ the set of all possible symbols
written in the scanned cell, when the machine state is a member of $A$ and the scanned cell a member
of $B$.

Henceforth, we shall refer to a set of states of $M$ as a \emph{stateset} and a set of tape symbols
of $M$ as a \emph{symbolset} respectively.
Statesets and symbolsets are closed under union.
Thus if $q_1, q_2$ are statesets, so is $q_1 + q_2$ and if $a_1, a_2$ are symbol subsets so is $a_1+a_2$.
The following lemma shows that distributive properties hold.
We use lower case symbols for statesets
and symbolsets.


\begin{lemma}\label{lem:distributive}
For any statesets $q, q_1, q_2$ and symbolsets $a, a_1, a_2$
\begin{align*}
    q*(a_1+a_2) &= q*a_1+q*a_2 \\
    (q_1 + q_2) * a &= q_1 * a + q_2 * a \\
    (a_1 + a_2) * q &= a_1 * q + a_2 * q \\
    a * (q_1 + q_2) &= a*q_1 + a*q_2
\end{align*}
\end{lemma}

\begin{proof}
    From the definition above,
    \begin{align*}
        p \in q * (a_1 + a_2) &\iff \exists p' \in q, x \in \Gamma, b \in a_1 + a_2 \suchthat (p,x) \in \delta(p', b) \\
        &\iff \exists p' \in q, x \in \Gamma, b \in a_1 \suchthat
        (p,x) \in \delta(p', b) \lor
        \exists p' \in q, x \in \Gamma, b \in a_2 \suchthat
        (p,x) \in \delta(p',b)\\
        &\iff p \in q * a_1 \lor p \in q * a_2 \\
        &\iff p \in q * a_1 + q * a_2
    \end{align*}
\end{proof}

\begin{aside}
    The original proof of Lemma~\ref{lem:distributive} is somewhat terse.
    However, the concept is fairly simple.
    From the definition of the product operator $*$, we have that
    $q * (a_1 + a_2)$ is the set of all states reachable from $q$, with
    a tape cell containing either $a_1$ or $a_2$.
    If $p$ is reachable from $q$ given a symbol from $a_1$ or $a_2$, then
    there must be an initial state $p' \in q$ such that we transition to
    $p$ with the symbol $b$ drawn from $a_1 + a_2$.
    We can then consider if the symbol $b$ comes from $a_1$ (in which case)
    $p \in q * a_1$, or if the symbol $b$ comes from $a_2$, in which case
    $p \in q * a_2$.
    If the symbol is common to both $a_1$ and $a_2$, the result is unaffected.
    As a result, we can claim that $p \in q*a_1 + q*a_2$.
\end{aside}

A stateset $q$ is \emph{left moving} if $q \subseteq Q^L$, \emph{right moving} if
$q \subseteq Q^R$ and halting if $q \cap H \neq \emptyset$.
\begin{error}
    Should this not be $q \subseteq Q^R \cup P^R$?
    The subset $P^R$ contains right moving states as well.
\end{error}
A symbolset $a$ is \emph{marked} if every tape symbol $x$ of $a$ contains the sweep marker.
The definitions of the instructions of $M$ show that a stateset of $M$ during any
computation is a left moving, right moving or a halting stateset.
The symbolset is either marked or unmarked.

Consider now the set of halting statesets $Q_H$.
If $M$ has been constructed to accept a recursive set $L$, then
for every input $w$ (encoded as defined subsequently), $M$ must halt
in an accepting or rejecting stateset.
Accordingly, we assume that $H = H_A \cup H_R$ where
$H_A$ is the set of accepting states and $H_R$ the set of
rejecting states.
$q \in Q_H$ is \emph{accepting} if $q \cap H_A \neq \emptyset$
and \emph{rejecting} if $q \cap H_A = \emptyset$.

We shall assume that there exist symbols $a_1, a_2 \in \Gamma$ such that
\[
    q \in H_A \implies \forall x \in \Gamma, \exists q' \in H_A \suchthat
    (q', a_1) \in \delta(q,x)
\]
and
\[
    q \in H_R \implies \forall x \in \Gamma, \exists q' \in H_R \suchthat
    (q', a_2) \in \delta(q,x).
\]
$a_1$ is therefore an accepting tapesymbol, and $a_2$ a rejecting
tapesymbol.  $H_A$ and $H_R$ would contain left moving and right moving states.

\begin{aside}
    The machine $M$ will write the symbol $a_1$ to the tape when it accepts the set $L$, and
    the machine $M$ will write the symbol $a_2$ to the tape when it rejects the set $L$.
\end{aside}

Let
\begin{align*}
    \Sigma_{A} &= \{a \subseteq \Gamma | a_1 \in a\} \\
    \Sigma_{R} &= \{a \subseteq \Gamma | a_2 \in a\} \\
    Q_A &= \{q \subseteq Q | q \cap H_A \neq \emptyset\} \\
    Q_R &= \{q \subseteq Q | q \cap H \neq \emptyset \land q \cap H \neq \emptyset \} \\
    \{Q\} &= \{q | q \subseteq Q^L \lor q \subseteq Q^R \}.
\end{align*}

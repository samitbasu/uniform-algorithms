

\iffalse
\begin{aside}
    For the engineers (like me) reading this, it can help to think in terms of a concrete
    machine, i.e., a simple computer.  In this case.  Let us assume that we have a
    computer with a very simple Harvard architecture:
    \begin{itemize}
        \item It has a single register/accumulator $a \in \mathbb{Z}^W$.
        \item It also has a single data register $x \in \mathbb{Z}^W$.
        \item It has a left, right, halt flag $f \in \{L, R, H\}$
        \item It has a set of opcodes $\{o_i\}$.
        \item It has a current instruction register $o$.
        \item It has an instruction pointer $n$.
        \item It has a data pointer $m$.
        \item It has a data memory space $T_k$ that is infinite and starts at index $k = 0$. Each
        $T_k \ in \mathbb{Z}^W$.
        \item It has a program memory space $D_k$ that contains opcodes $D_k \in \{o_i\}$.
    \end{itemize}

    It is easy to reason about the behavior of this simple computer.  The architecture is
    thus:
    \begin{itemize}
        \item The computer is initialized with $a = 0, x = 0, n = 0, m = 0$, and some contents of the data memory $T_k$,
        and the program memory $D_k$.
        \item On each "clock cycle", the computer performs the following sequence of events:
        \begin{enumerate}
            \item Fetch the contents of the program memory at the current instruction pointer into $o$.
            \item Fetch the contents of the data memory at the current data pointer into $x$.
            \item Based on the instruction $o$, use the value of the accumulation register, the opcode, and the data element, to compute updated values for these:

            update $a, n, m, x$.
            \item If the opcode is \verb|HALT|, the computer stops.
            \item Otherwise, it continues with the first step.
        \end{enumerate}
    \end{itemize}

    This computer (while limited) contains the following equivalencies when describing
    the components of the \emph{S-machine}:
    \begin{itemize}
        \item The set $Q$ is the set of states of the machine.  In this case, the state of the computer
        would technically include the contents of the memory (or in Turing machine language, the contents
        of the tape).  But in practice, we think instead of the state of the processor itself.  For this
        processor, the state is simply $<a, n, m>$
    \end{itemize}
\end{aside}

\begin{tikzpicture}[every node/.style={block},
    block/.style={minimum height=3.5em,outer sep=0pt,draw,rectangle,node distance=0pt}]
    \node (A) {$\substack{x\\\cancelb}$};
    \node (B) [left=of A] {$\ldots$};
    \node (C) [left=of B] {$B$};
    \node (D) [right=of A] {$\ldots$};
    \node (E) [right=of D] {$\$ $};
    \node (F) [above = 0.75cm of A,draw=red,thick] {$q_0$};
    \draw[-latex] (F) -- (A);
    \draw[-latex,blue] ($(F.east)!0.5!(A.east)$) -- ++(7mm,0);
    \draw (C.north west) -- ++(-1cm,0) (C.south west) -- ++ (-1cm,0)
    (E.north east) -- ++(1cm,0) (E.south east) -- ++ (1cm,0);
\end{tikzpicture}
\fi